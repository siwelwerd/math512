\documentclass{article}

\usepackage[top=0.75in,bottom=1in,left=1in,right=1in]{geometry}
\usepackage{enumerate,hyperref,amssymb,caption,amsmath}

\newcommand{\prof}{Dr. Drew Lewis}
\newcommand{\profemail}{{\tt drewlewis@southalabama.edu}}
\newcommand{\profoffice}{306 MSPB}
\newcommand{\profhours}{M 12-3:15 W 12-2:45 }
\newcommand{\course}{Math 512}
\newcommand{\semester}{Spring 2022 }
\newcommand{\classtime}{TR 4:30--5:45}
\newcommand{\LMS}{Canvas}



\begin{document}



\begin{center}
{\bf \large \course{}\\
Algebra II} \\
\end{center}

\vspace{0.25in}


\noindent \begin{tabular}{@{}ll}
\begin{tabular}{ll}
{\bf Course information:}& \course{} -- Algebra II \\
& \semester \\
& Course format: Web-enhanced \\
& \\
{\bf Meeting times:} & \classtime \end{tabular} &
\begin{tabular}{ll}
{\bf Instructor:}& \prof \\
& \profemail \\
& {\tt(251) 341-3094} \\
& \\
{\bf Office hours:} & \profhours
\end{tabular}
\end{tabular}

\section*{\fontsize{12}{15}\selectfont Student Success}
I am committed to helping each and every one of you achieve your goals. I fundamentally believe that \textbf{everyone is a math person} and is capable of succeeding in this course. If there is something we could do differently that will help you succeed, please come talk to me. 

There are many services on campus to help you succeed. I enthusiastically support the mission of our Office of Student Disability Services; if you are registered with them, please come speak to me in office hours so I can ensure I am doing everything needed to enable your success.


\section*{\fontsize{12}{15}\selectfont Basic Needs Security}
Any student who has difficulty affording groceries or accessing sufficient food to eat every day, or who lacks a safe and stable place to live, and believes this may affect their performance in the course, is urged to contact the Dean of Students for support. Furthermore, please notify the professor if you are comfortable in doing so. This will enable me to try and provide information on accessing additional resources.  Information on the campus food pantry is available at \url{https://southalabama.edu/departments/sga/foodpantry.html}.


\section*{\fontsize{12}{15}\selectfont Communication Plan}

The best way to reach me is by email at {\tt drewlewis@southalabama.edu}. Alternatively, stop by my office with quick questions. For longer questions, please use \href{https://calendly.com/dr-lewis}{this link} to schedule an appointment.

\section*{\fontsize{12}{15}\selectfont Office Hours and Instructor Availability}

Scheduled office hours are listed at the top of the syllabus. For appointments at other times, please use \href{https://calendly.com/dr-lewis}{this link} to schedule an appointment--it links to my calendar to show all my availability. 

\section*{\fontsize{12}{15}\selectfont Disaster Plan}
In the event of a campus closure (e.g. due to a pandemic or weather), we will continue to meet virtually via Zoom at our regular time as best we can. In such an event, I ask that we all extend grace and patience to each other; we will all do the best we can in the circumstances.


\section*{\fontsize{12}{15}\selectfont Course Description}
A graduate level introduction to ring theory and fields. Topics include ring homomorphisms, quotient rings, ideals, rings of fractions, Euclidean domains, principal ideal domains, unique factorization domains, modules, finite fields, field extensions.


\section*{\fontsize{12}{15}\selectfont Assignments}
\begin{itemize}
\item \textbf{Problem sets: } Approximately weekly, you will submit problem sets through Canvas.  I will provide feedback, and then you can resubmit a revised version. We can repeat this cycle as many times as needed. To facilitate revisions, I encourage you to typeset your solutions in \LaTeX.
\item \textbf{Final: } We will have a final exam due during the registrar assigned time.
\item \textbf{Final portfolio: } At the end of the semester, you will submit a final portfolio consisting of all of your (revised) problem sets, as well as completed (revised) midterm and final exams.  This will be submitted through Canvas as well.
\end{itemize}


\section*{\fontsize{12}{15}\selectfont Grading}
Course grades will be based on your final exam and the final portfolio. Obviously, the key to a complete portfolio is to complete the weekly problem sets along the way, incorporate feedback, and revise them.

\section*{\fontsize{12}{15}\selectfont Collaboration}
Mathematics is a very collaborative discipline.   For assignments, use the following guidelines:
\begin{itemize}
\item On \textbf{problem sets}, I would like you to try all problems alone initially. If you get stuck, set it aside and come back to it later.  If you are still stuck, then come talk to me or with your classmates.  On each problem set you submit, you will include an `acknowledgements' section where you acknowledge who helped you on which problems, and the degree of that help.
\item The \textbf{final portfolio} will be assembling items from the previous categories, so should not require any significant collaboration.
\end{itemize}



\section*{\fontsize{12}{15}\selectfont Student Academic Conduct Policy}
All students are expected to adhere to the Student Academic Conduct Policy, which you can view at
{\tt http://www.southalabama.edu/bulletin/current/student-affairs/conduct.html}.  Students violating this policy will be given one or more of the following penalties based on the severity of the offense:  1) Reduction in final course grade by a letter grade; 2) Automatic course failure.


\section*{\fontsize{12}{15}\selectfont Syllabus Changes}
We may want to make changes to this syllabus due to the development of the pandemic, weather events like hurricanes, etc. Any changes made will reflect the spirit of this original syllabus, and will be updated on \LMS.




\end{document}
