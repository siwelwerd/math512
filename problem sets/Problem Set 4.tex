\documentclass{problemset}
\usepackage{amscd}

\renewcommand{\pset}{4}
%\renewcommand{\name}{Jane Doe}

\DeclareMathOperator{\Frac}{Frac}
\DeclareMathOperator{\im}{Im}
\DeclareMathOperator{\Hom}{Hom}

\begin{document}

\begin{exercise}If \(A\) is a finite abelian group, show that \(A \otimes _{\mathbb{Z}} \mathbb{Q} = 0\).
\end{exercise}

%\begin{solution}
%Solution goes here
%\end{solution}

\begin{exercise} Show that \(\mathbb{Z}_m \otimes _{\mathbb{Z}} \mathbb{Z}_n \cong \mathbb{Z}_d\), where \(d=(m,n)\).  {\bf Hint: } Write \(d=am+bn\) for some integers \(a,b\). 
\end{exercise}

%\begin{solution}
%Solution goes here
%\end{solution}



\begin{exercise} Let \(M\) be an \(R\)-module, and \(I \subset R\) an ideal.  Show that \(R/I) \otimes _R M \cong M/IM\).
\end{exercise}

%\begin{solution}
%Solution goes here
%\end{solution}

\begin{exercise} Let \(R\) be commutative, and \(I,J \subset R\) ideals.  Show that \(R/I \otimes _R R/J \cong R/(I+J)\).
\end{exercise}

%\begin{solution}
%Solution goes here
%\end{solution}

\begin{exercise} Let \(R\) be commutative.  An \(R\)-module is called {\em flat} if tensoring with that module is left exact.  Show that every projective \(R\)-module is flat.
\end{exercise}

%\begin{solution}
%Solution goes here
%\end{solution}






%\begin{acknowledgements}
% Acknowledge your collaborators here. Be specific about who helped on which problems, and for which parts.  For example, ``Jane helped me on Exercise 3. I got stuck proving X, but once she showed me the trick, I was able to finish.''
%\end{acknowledgements}

\end{document}
