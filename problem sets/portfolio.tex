\documentclass{problemset}
\usepackage{titlesec,hyperref,amscd}

\renewenvironment{exercise}{\refstepcounter{exercise}
   \noindent \textbf{Exercise~\thesection.\theexercise. }}{ \vskip-\lastskip \vspace{0.2in} }
   
\titleformat{\section}
{\normalfont}
{}
{8pt}
{\Large\bfseries\filcenter}

\hypersetup{
  colorlinks   = true, %Colours links instead of ugly boxes
  urlcolor     = blue, %Colour for external hyperlinks
  linkcolor    = blue, %Colour of internal links
  citecolor   = red %Colour of citations
}


\DeclareMathOperator{\Frac}{Frac}
\DeclareMathOperator{\im}{Im}
\DeclareMathOperator{\Hom}{Hom}

\DeclareMathOperator{\rad}{rad}

\chead{Portfolio}
%\renewcommand{\name}{Jane Doe}

\begin{document}

\tableofcontents
\newpage


\section{Problem Set 1}


\begin{exercise} Let \(R\) be an integral domain, and let \(T\) be an integral domain such that \(R \subset T \subset \Frac R\).  Show that \(\Frac R = \Frac T\).

\end{exercise}

%\begin{solution}
%Solution goes here
%\end{solution}

\begin{exercise}
Let \(R\) be an integral domain, and \(S \subset R\) a multiplicative subset that does not contain \(0\). Show that if \(R\) is a PID, then so is \(S^{-1}R\).
\end{exercise}

%\begin{solution}
%Solution goes here
%\end{solution}

\begin{exercise}
Let \(R\) be a commutative unital ring, \(S \subset R\) a multiplicative subset, and \(I \subset R\) an ideal.  Show that \(S^{-1}\sqrt{I}=\sqrt{S^{-1}I}\) (Recall that \(\sqrt{I}=\left\{x \in R\ |\ x^n \in I\ \text{for some }n \in \mathbb{N}\right\}\)).
\end{exercise}

%\begin{solution}
%Solution goes here
%\end{solution}


\begin{exercise}
Let \(R\) be a commutative unital ring.  Show that \(R\) is local if and only if whenever \(r+s=1\), then either \(r \in R^*\) or \(s \in R^*\)
\end{exercise}

%\begin{solution}
%Solution goes here
%\end{solution}

\begin{exercise}
Show that every nonzero homomorphic image of a local ring is local.
\end{exercise}

%\begin{solution}
%Solution goes here
%\end{solution}

\newpage
\section{Problem Set 2}

\begin{exercise} Let \(R\) be a commutative unital ring.  Show that \[M_n = \left\{ p \in R[x]\ \middle|\ \deg p < n \right\}\] is a submodule of \(R[x]\).
\end{exercise}

%\begin{solution}
%Solution goes here
%\end{solution}

\begin{exercise} Let \(M\) be an \(R\)-module, and \(I \subset R\) an ideal. 
\begin{enumerate}
\item Show that \(IM = \{ \sum _{i=1} ^n r_i m_i\ |\ n \in \mathbb{N}, r_i \in I, m_i \in M\}\) is a submodule of \(M\).
\item Show that \(M/IM\) is an \(R/I\) module, with multiplication given by \[(r+I)(m+IM)=rm+IM\ \text{for all }r +I\in R/I, m+IM \in M/IM.\]
\end{enumerate}
\end{exercise}

%\begin{solution}
%Solution goes here
%\end{solution}

\begin{exercise} Prove the Five Lemma: Let
\[\begin{CD}
A_1 @>>> A_2 @>>> A_3 @>>> A_4 @>>> A_5\\
@VV\alpha _1V @VV\alpha _2V @VV\alpha _3V @VV\alpha _4V @VV\alpha _5V \\
B_1 @>>> B_2 @>>> B_3 @>>> B_4 @>>> B_5\\
\end{CD}\]
be a commutative diagram of \(R\)-module homomorphisms with each row exact.
\begin{enumerate}[(a)]
\item Show that if \(\alpha _1\) is surjective, and \(\alpha _2\) and \(\alpha _4\) are injective, then \(\alpha _3\) is also injective.
\item Show that if \(\alpha _5\) is injective, and \(\alpha _2\) and \(\alpha _4\) are surjective, then \(\alpha _3\) is also surjective.
\end{enumerate}
\end{exercise}

%\begin{solution}
%Solution goes here
%\end{solution}



\begin{exercise} 
Let \(f: A \to A\) be an \(R\)-module homomorphism.  Show that if \(ff=f\), then \(A \cong \ker f \oplus \im f\).
\end{exercise}

%\begin{solution}
%Solution goes here
%\end{solution}

\begin{exercise} Let \(f: A \to B\) and \(g: B \to A\) be \(R\)-module homomorphisms.  Show that if \(gf={\rm id}\), then \(B \cong \im f \oplus \ker g\).
\end{exercise}

%\begin{solution}
%Solution goes here
%\end{solution}

\newpage
\section{Problem Set 3}

\begin{exercise} Let \(R\) be a ring, and \(M\) an abelian group.  Define \[\Hom _{\mathbb{Z}}(R,M)=\left\{f: R \rightarrow M\ |\ f\ \text{is a }\mathbb{Z}\text{-module homomorphism}\right\}.\]
Show that \(\Hom _{\mathbb{Z}}(R,M)\) is an \(R\)-module with multiplication \((rf)(x)=rf(x)\) for any \(r \in R\), \(f \in \Hom _{\mathbb{Z}}(R,M)\), and \(x \in R\).
\end{exercise}

%\begin{solution}
%Solution goes here
%\end{solution}

\begin{exercise} Show that \(\mathbb{Q}\) is not a projective \(\mathbb{Z}\)-module.
\end{exercise}

%\begin{solution}
%Solution goes here
%\end{solution}

\begin{exercise} Show that every projective abelian group is free.
\end{exercise}

%\begin{solution}
%Solution goes here
%\end{solution}

\begin{exercise} Show that a direct product of \(R\)-modules \(\prod _{i \in I} J_i\)  is injective if and only if each \(J_i\) is injective.
\end{exercise}

%\begin{solution}
%Solution goes here
%\end{solution}





\begin{exercise} Let \(R\) be a commutative, unital ring.  Show that the following are equivalent.
\begin{enumerate}[(i)]
\item Every \(R\)-module is projective.
\item Every \(R\)-module is injective.
\item Every short exact sequence of \(R\)-modules is split exact.
\end{enumerate}
\end{exercise}

%\begin{solution}
%Solution goes here
%\end{solution}

\newpage
\section{Problem Set 4}

\begin{exercise}If \(A\) is a finite abelian group, show that \(A \otimes _{\mathbb{Z}} \mathbb{Q} = 0\).
\end{exercise}

%\begin{solution}
%Solution goes here
%\end{solution}

\begin{exercise} Show that \(\mathbb{Z}_m \otimes _{\mathbb{Z}} \mathbb{Z}_n \cong \mathbb{Z}_d\), where \(d=(m,n)\).  {\bf Hint: } Write \(d=am+bn\) for some integers \(a,b\). 
\end{exercise}

%\begin{solution}
%Solution goes here
%\end{solution}



\begin{exercise} Let \(M\) be an \(R\)-module, and \(I \subset R\) an ideal.  Show that \(R/I) \otimes _R M \cong M/IM\).
\end{exercise}

%\begin{solution}
%Solution goes here
%\end{solution}

\begin{exercise} Let \(R\) be commutative, and \(I,J \subset R\) ideals.  Show that \(R/I \otimes _R R/J \cong R/(I+J)\).
\end{exercise}

%\begin{solution}
%Solution goes here
%\end{solution}

\begin{exercise} Let \(R\) be commutative.  An \(R\)-module is called {\em flat} if tensoring with that module is left exact.  Show that every projective \(R\)-module is flat.
\end{exercise}

%\begin{solution}
%Solution goes here
%\end{solution}


\newpage
\section{Problem Set 5}

\begin{exercise}Let \(R\) be a PID, and \(I \subset R\) an ideal.  Show that \(R/I\) is both Noetherian and Artinian.
\end{exercise}

%\begin{solution}
%Solution goes here
%\end{solution}

\begin{exercise}Let \(R\) be Noetherian, and \(P \subset R\) a prime ideal.  Show that \(R_P\) is Noetherian.
\end{exercise}

%\begin{solution}
%Solution goes here
%\end{solution}


\begin{exercise}
Let \(R\) be an Artinian ring.  Show that every prime ideal of \(R\) is maximal.
\end{exercise}

%\begin{solution}
%Solution goes here
%\end{solution}

\begin{exercise}Let \(R\) be a ring, \(S \subset R\) a multiplicative set, and \(I \subset R\) an ideal.  Show that \(S^{-1}(\rad I) = \rad(S^{-1}I)\).
\end{exercise}

%\begin{solution}
%Solution goes here
%\end{solution}

\begin{exercise}Let \(R\) be Noetherian, and \(I,J \subset R\) ideals with \(J \subset \rad I\).  Show that there exists \(n \in \mathbb{N}\) with \(J^n \subset I\).
\end{exercise}

%\begin{solution}
%Solution goes here
%\end{solution}



\newpage
\section{Problem Set 6}

\begin{exercise}Let \(R\) be a ring in which every maximal ideal is of the form \(cR\) for some \(c \in R\) satisfying \(c^2=c\).  Show that \(R\) is Noetherian (\textbf{Hint:} Show that every prime ideal is maximal).
\end{exercise}

%\begin{solution}
%Solution goes here
%\end{solution}

\begin{exercise}Let \((R, \mathfrak{M})\) be a Noetherian local ring.  Suppose that \(\mathfrak{M}/\mathfrak{M}^2\) is generated by the set \(\{a_1+\mathfrak{M}^2,\ldots,a_n+\mathfrak{M}^2\}\).  Show that \(\mathfrak{M}=a_1R+\cdots+a_nR\).
\end{exercise}

%\begin{solution}
%Solution goes here
%\end{solution}

\begin{exercise}Let \(R \subset S\) be an integral extension, and suppose that \(R\) and \(S\) are both integral domains.  Show that \(R\) is a field if and only if \(S\) is a field.
\end{exercise}

%\begin{solution}
%Solution goes here
%\end{solution}

\begin{exercise}Show that if \(R \subset S\) is an integral extension, then \(S[x_1,\ldots,x_n]\) is integral over \(R[x_1,\ldots,x_n]\).
\end{exercise}

%\begin{solution}
%Solution goes here
%\end{solution}

\begin{exercise}Let \(R\) be an integral domain with fractional field \(k\).  Show that if \(R\) is integrally closed and \(t\) is transcendental over \(k\), then \(R[t]\) is integrally closed.
\end{exercise}

%\begin{solution}
%Solution goes here
%\end{solution}






\newpage
\begin{acknowledgements}
% Acknowledge your collaborators here. Be specific about who helped on which problems, and for which parts.  For example, ``Jane helped me on Exercise 3. I got stuck proving X, but once she showed me the trick, I was able to finish.''
\end{acknowledgements}

\end{document}
