\documentclass{problemset}
\usepackage{amscd}

\renewcommand{\pset}{2}
%\renewcommand{\name}{Jane Doe}

\DeclareMathOperator{\Frac}{Frac}
\DeclareMathOperator{\im}{Im}

\begin{document}

\begin{exercise} Let \(R\) be a commutative unital ring.  Show that \[M_n = \left\{ p \in R[x]\ \middle|\ \deg p < n \right\}\] is a submodule of \(R[x]\).
\end{exercise}

%\begin{solution}
%Solution goes here
%\end{solution}


\begin{exercise} Prove the Five Lemma: Let
\[\begin{CD}
A_1 @>>> A_2 @>>> A_3 @>>> A_4 @>>> A_5\\
@VV\alpha _1V @VV\alpha _2V @VV\alpha _3V @VV\alpha _4V @VV\alpha _5V \\
B_1 @>>> B_2 @>>> B_3 @>>> B_4 @>>> B_5\\
\end{CD}\]
be a commutative diagram of \(R\)-module homomorphisms.
\begin{enumerate}[(a)]
\item Show that if \(\alpha _1\) is surjective, and \(\alpha _2\) and \(\alpha _4\) are injective, then \(\alpha _3\) is also injective.
\item Show that if \(\alpha _5\) is injective, and \(\alpha _2\) and \(\alpha _4\) are surjective, then \(\alpha _3\) is also surjective.
\end{enumerate}
\end{exercise}

%\begin{solution}
%Solution goes here
%\end{solution}



\begin{exercise} 
Let \(f: A \to A\) be an \(R\)-module homomorphism.  Show that if \(ff=f\), then \(A \cong \ker f \oplus \im f\).
\end{exercise}

%\begin{solution}
%Solution goes here
%\end{solution}

\begin{exercise} Let \(f: A \to B\) and \(g: B \to A\) be \(R\)-module homomorphisms.  Show that if \(gf={\rm id}\), then \(B \cong \im f \oplus \ker g\).
\end{exercise}

%\begin{solution}
%Solution goes here
%\end{solution}

\begin{exercise} 
\end{exercise}

%\begin{solution}
%Solution goes here
%\end{solution}


%\begin{acknowledgements}
% Acknowledge your collaborators here. Be specific about who helped on which problems, and for which parts.  For example, ``Jane helped me on Exercise 3. I got stuck proving X, but once she showed me the trick, I was able to finish.''
%\end{acknowledgements}

\end{document}
