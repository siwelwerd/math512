\documentclass{problemset}


\renewcommand{\pset}{1}
%\renewcommand{\name}{Jane Doe}

\DeclareMathOperator{\Frac}{Frac}
\DeclareMathOperator{\im}{Im}

\begin{document}


\begin{exercise} 
\end{exercise}

%\begin{solution}
%Solution goes here
%\end{solution}

\begin{exercise} 
\end{exercise}

%\begin{solution}
%Solution goes here
%\end{solution}

\begin{exercise} 
Let \(f: A \to A\) be an \(R\)-module homomorphism.  Show that if \(ff=f\), then \(A \cong \ker f \oplus \im f\).
\end{exercise}

%\begin{solution}
%Solution goes here
%\end{solution}

\begin{exercise} Let \(f: A \to B\) and \(g: B \to A\) be \(R\)-module homomorphisms.  Show that if \(gf={\rm id}\), then \(B \cong \im f \oplus \ker g\).
\end{exercise}

%\begin{solution}
%Solution goes here
%\end{solution}

\begin{exercise} 
\end{exercise}

%\begin{solution}
%Solution goes here
%\end{solution}


%\begin{acknowledgements}
% Acknowledge your collaborators here. Be specific about who helped on which problems, and for which parts.  For example, ``Jane helped me on Exercise 3. I got stuck proving X, but once she showed me the trick, I was able to finish.''
%\end{acknowledgements}

\end{document}
