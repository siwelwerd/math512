\documentclass{problemset}

\renewcommand{\pset}{1}
%\renewcommand{\name}{Jane Doe}

\begin{document}


\begin{exercise} Show the following directly from the definitions:
\begin{enumerate}[(a)]
\item If \(G\) is a monoid, the identity element is unique. 
\item If \(G\) is a group, every element has a unique inverse.
\end{enumerate}
\end{exercise}

%\begin{solution}
%Solution goes here
%\end{solution}

\begin{exercise}
Let \(G\) be a semigroup. Show that \(G\) is a group if and only if for all \(a,b \in G\), the equations \(ax=b\) and \(ya=b\) have solutions in \(G\).
\end{exercise}

%\begin{solution}
%Solution goes here
%\end{solution}

\begin{exercise}
Let \(G\) be a group.  Show that the following are equivalent:
\begin{enumerate}[(i)]
\item \(G\) is abelian.
\item \((ab)^2=a^2b^2\) for all \(a,b \in G\).
\item \((ab)^{-1}=a^{-1}b^{-1}\) for all \(a,b \in G\).
\item \((ab)^n=a^nb^n\) for three consecutive integers \(n\) and for all \(a,b \in G\).
\end{enumerate}
\end{exercise}

%\begin{solution}
%Solution goes here
%\end{solution}


\begin{exercise}
Let \(G\) be a semigroup. \(G\) is called \term{left cancellative} if \(ab=ac\) implies \(b=c\) (for all \(a,b,c \in G\)), and is called \term{right cancellative} if \(ba=ca\) implies \(b=c\) (again for all \(a,b,c \in G\).  A semigroup that is both left and right cancellative is just called \term{cancellative}. 
\begin{enumerate}[(a)]
\item Show that every finite cancellative semigroup is a group.
\item Give an example of an infinite cancellative semigroup that is not a group.
\end{enumerate}
\end{exercise}

%\begin{solution}
%Solution goes here
%\end{solution}

\begin{exercise}
Let \(G\) be a cyclic group. Show that every homomorphic image of \(G\) and every subgroup of \(G\) is also cyclic.
\end{exercise}

%\begin{solution}
%Solution goes here
%\end{solution}

\begin{exercise}
Let \(G\) be an abelian group of order \(pq\) for coprime \(p,q\in \mathbb{N}\).  Show that if \(G\) contains elements of both order \(p\) and order \(q\), then \(G\) must be cyclic.
\end{exercise}

%\begin{solution}
%Solution goes here
%\end{solution}

\begin{exercise}
Let \(f:G \rightarrow H\) be a group homomorphism.  Suppose \(a\in G\) such that \(f(a) \in H\) has finite order.  Show that if \(|a|\) is finite,  then \(|f(a)|\) divides \( |a|\).
\end{exercise}

%\begin{solution}
%Solution goes here
%\end{solution}

\begin{exercise}
Show that every group with a finite number of subgroups is itself finite.
\end{exercise}

%\begin{solution}
%Solution goes here
%\end{solution}


\begin{exercise}
Let \(H,K\) be subgroups of a group \(G\). Show that \(HK\) is a subgroup if and only if \(HK=KH\).
\end{exercise}

%\begin{solution}
%Solution goes here
%\end{solution}

%\begin{acknowledgements}
% Acknowledge your collaborators here. Be specific about who helped on which problems, and for whic parts.  For example, ``Jane helped me on Exercise 3. I got stuck proving X, but once she showed me the trick, I was able to finish.''
%\end{acknowledgements}

\end{document}